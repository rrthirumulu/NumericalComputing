\documentclass{article}

% Packages
\usepackage{amsmath, amssymb}   % math formatting & symbols
\usepackage{graphicx}           % insert graphics
\usepackage{listings}
%\usepackage{eulervm, bookman}   % fonts for math & symbols
\usepackage{fullpage}          % fullpage margins

\lstset{language=Matlab,
    basicstyle=\scriptsize\ttfamily,
    stringstyle=\ttfamily,
    showstringspaces=false,
    breaklines=true,
    frameround=ffff,
    frame=single,
}

\begin{document}
% Title 
\title{Math 104A: Homework 2}
\author{Raghav Thirumulu, Perm 3499720 \\ \texttt{rrajuthirumulu@umail.ucsb.edu}}
\maketitle

\begin{enumerate}
    \item % 1 
        \begin{enumerate}
            \item % 1a
                We must construct the basis polynomials first.
                \begin{align*}
                    L_{2,0}(x)    &= \frac{(x-x_{1})(x-x_{2})}{(x_{0}-x_{1})(x_{0}-x_{2})} =                           \frac{(x-1)(x-3)}{(0-1)(0-3)} =
                                     \frac{x^2-4x+3}{3}   \\
                    L_{2,1}(x)    &= \frac{(x-x_{0})(x-x_{2})}{(x_{1}-x_{0})(x_{1}-x_{2})} =                           \frac{(x-0)(x-3)}{(1-0)(1-3)} =
                                     \frac{x^2-3x}{-2}   \\
                    L_{2,2}(x)    &= \frac{(x-x_{0})(x-x_{1})}{(x_{2}-x_{0})(x_{2}-x_{1})} =                           \frac{(x-0)(x-1)}{(3-0)(3-1)} =
                                     \frac{x^2-x}{6}   \\
                \end{align*}    
                Now use the Lagrange interpolation formula to find \textit{P}\textsubscript{2}(\textit{x})
                \begin{align*}
                    P_{2}(x)  & = f(x_{0})L_{2,0}(x) + f(x_{1})L_{2,1}(x) + f(x_{2})L_{2,2}(x)\\
                              & = 1(\frac{x^2-4x+3}{3}) + 1(\frac{x^2-3x}{-2}) - 5(\frac{x^2-x}{6})\\
                              & = -x^2 + x + 1\\
                \end{align*}
            \item % 1b
                Solve for \textit{P}\textsubscript{2}(\textit{x}) where \textit{x} = 2
                \begin{align*}
                    P_{2}(2)  & = -(2)^2 + 2 + 1 = -1\\
                \end{align*}
        \end{enumerate}
    \item % 2 
        \begin{enumerate}
            \item % 2a
                \begin{lstlisting}
                function T = lebesgue(x,xbar) 
                % Computer code for evaluating Lebesque function
                % Input:  x    --- nodes for interpolation;
                %         xbar --- point for evaluation;
                % Output: T    --- the evaluation of the Lebesgue function
                %
                % Author: Raghav Thirumulu, Perm 3499720
                % Date:   07/10/2018
                
                % Store length of x in n, Create 1xn array of ones for storing values
                n=length(x);
                L=ones(1,n);

                % Iterate through the interpolation points
                for i=1:n
                    for j=1:n
                        if j~=i
                            L(i) = L(i) * ( (xbar-x(j)) / (x(j)-x(i)) ) ;
                        end
                    end
                end

                % Take the absolute value of the sum of the elements of L
                T = sumabs(L);

                end
                
                \end{lstlisting}
            
            \item % 2b 
                \begin{lstlisting}
                function F = lebesgue_run(n)
                % Computer code for calling lebesque.m and finding the constant
                % based on number of evaluation points
                % Input:  n    --- number of evaluation points;
                % Output: F    --- the Lebesgue constant
                %
                % Author: Raghav Thirumulu, Perm 3499720
                % Date:   07/10/2018

                % Create a row vector for storing interpolation points
                x=zeros(1,n+1);

                % Find interpolation points and store in x using given formula
                for j=1:n+1
                    x(j)=-1+(j-1)*(2/n);
                end

                % Adjust parameter for plotting resolution
                m=1000;
                T=zeros(1,m+1);
                xbar=zeros(1,m+1);

                % Find points for xbar through iteration with plotting resolution
                for k=1:m+1
                    xbar(k)=-1+(k-1)*(2/m);
                    T(k)=lebesgue(x,xbar(k));
                end

                % Plot the Lebesgue function
                plot(xbar,T);
                xlabel('x');
                ylabel('L(x)');
                title(['n = ' num2str(n)]);

                % Find the norm using the data, F will be Lebesgue constant
                F=norm(T,Inf);
                end
                \end{lstlisting}
                
                \begin{center}
                    \includegraphics[scale=0.6]{n4.png}
                    \includegraphics[scale=0.6]{n10.png}
                    \includegraphics[scale=0.6]{n20.png}
                \end{center}
                
                The Lebesgue constant for n=4 is 2.2078 \\
                The Lebesgue constant for n=10 is 29.8981 \\
                The Lebesgue constant for n=20 is 10979 \\
                
                \item % 2b 
                \begin{lstlisting}
                function F = lebesgue_run2(n)
                % Computer code for calling lebesque.m and finding the constant
                % based on number of evaluation points
                % Input:  n    --- number of evaluation points;
                % Output: F    --- the Lebesgue constant
                %
                % Author: Raghav Thirumulu, Perm 3499720
                % Date:   07/10/2018

                % Create a row vector for storing interpolation points
                x=zeros(1,n+1);

                % Find interpolation points and store in x using given formula
                for j=1:n+1
                    x(j)=cos((j-1)*pi/n);
                end

                % Adjust parameter for plotting resolution
                m=1000;
                T=zeros(1,m+1);
                xbar=zeros(1,m+1);

                % Find points for xbar through iteration with plotting resolution
                for k=1:m+1
                    xbar(k)=-1+(k-1)*(2/m);
                    T(k)=lebesgue(x,xbar(k));
                end

                % Plot the Lebesgue function
                plot(xbar,T);
                xlabel('x');
                ylabel('L(x)');
                title(['n = ' num2str(n)]);

                % Find the norm using the data, F will be Lebesgue constant
                F=norm(T,Inf);
                end
                \end{lstlisting}
                
                \begin{center}
                    \includegraphics[scale=0.6]{n4_2.png}
                    \includegraphics[scale=0.6]{n10_2.png}
                    \includegraphics[scale=0.6]{n20_2.png}
                \end{center}
                
                The Lebesgue constant for n=4 is 1.7988 \\
                The Lebesgue constant for n=10 is 2.4210 \\
                The Lebesgue constant for n=20 is 2.8677 \\
                
                The Lebesgue function fluctuates more as n increases with Chebyshev points rather than equidistributed points. The Lebesgue constant also increases faster with equidistributed points than it does with Chebyshev points.
            
            
        \end{enumerate}
    \item % 3
    \begin{enumerate}
    \item 
        \begin{lstlisting}
        function T = barycentric_weights(x)
        % Computer code for finding the weights of the Barycentric Formula
        % Input:  x --- input points;
        % Output: T --- vector of weights to plug into barycentric.m
        %
        % Author: Raghav Thirumulu, Perm 3499720
        % Date:   07/11/2018

        % Find length of our input list, create another vector to substitute 
        % weights
        n=length(x);
        T=zeros(1,n);

        % Iterate through the input list and with interpolation, use the formula 
        % for calculating the weights
        for j=1:n
            w=1;
            for k=1:n
                if j~=k
            w = w*(x(j)-x(k));
            end
        end
        T(j)=w; 
        end

        % We want the reciprocal of the calculation of w
        T=1./T;

        end
        \end{lstlisting}
        
        \begin{lstlisting}
        function T = barycentric(x,y,w,z)
        % Computer code for implmenting Barycentric Formula
        % Input:  x --- sample input points
        %         y --- sample output points
        %         w --- weights calculated through barycentric_weights.m
        %         z --- point we are trying to approximate  
        % Output: T --- approximation for f(x) using Barycentric Formula
        % Author: Raghav Thirumulu, Perm 3499720
        % Date:   07/11/2018

        % Find length of the lists we are working with, create vector for
        % approximating Barycentric
        n=length(x);
        m=length(z);
        T=zeros(1,m);

        % Iterate through the lists, finding the interpolation values at each node
        for i=1:m
            k=0;
            num=0;
            den=0;
            for j=1:n
                if z(i)==x(j)
                    T(i)=y(j);
                    k=1;
                else
                    % Solve for numerator and denomintor of Barycentric formula
                    num=num+(w(j)*y(j))/(z(i)-x(j));
                    den=den+(w(j)/(z(i)-x(j)));
                end
            end
            if k==0
                T(i)=num/den;
            end
        end
        \end{lstlisting}
        
        \item
        \begin{lstlisting}
        function T = barycentric_run(x,y,z)
        % Computer code for applying Barycentric Formula
        % Input:  x --- sample input points
        %         y --- sample output points
        %         z --- point we are trying to approximate  
        % Output: T --- approximation for f(x) using Barycentric Formula
        % Author: Raghav Thirumulu, Perm 3499720
        % Date:   07/11/2018

        % Calculate weights using barycentric_weights.m
        w=barycentric_weights(x);

        % Pass calculated weights along with sample points to barycentric
        p=barycentric(x,y,w,z);
        disp(p);

        end
        \end{lstlisting}
        
        Running the function above with the column \textit{x\textsubscript{j}} stored in a vector called \textit{x} and column \textit{f}(\textit{x\textsubscript{j}}) stored in a vector called \textit{y}, along with \textit{z}=2 gives us an approximation of \textit{f}(2)=0.8520
        
        
        
    \end{enumerate}
    \item % 4 
        \begin{enumerate}
            \item 
            \begin{lstlisting}
            function T = runge(n)
            % Computer code for finding the weights of the Barycentric Formula
            % Input:  x --- input points;
            % Output: T --- vector of weights to plug into barycentric.m
            %
            % Author: Raghav Thirumulu, Perm 3499720
            % Date:   07/11/2018

            % Create row vectors based on n or number of nodes
            x=zeros(1,n+1);
            y=zeros(1,n+1);
            w=zeros(1,n+1);

            % Iterate through the function, finding the x and y values based on the
            % given formula
            for j=1:n+1
                x(j)=-5+(j-1)*(10/n);
                y(j)=1/(1+(x(j))^2);
            end

            % For equidistributed nodes we can use binomial coefficient
            for i=1:n+1
                w(i)=((-1)^(i-1))*nchoosek(n,i-1);
            end

            % We now will create a function with a very high resolution
            m=1000;
            z=zeros(1,m+1);
            F=zeros(1,m+1);

            % Iterate through the formula given for larger number m
            for k=1:m+1
                z(k)=-5+(k-1)*(10/m);
                F(k)=1/(1+(z(k))^2);
            end

            % Plot approximation versus actual function to visualize accuracy
            T = barycentric(x,y,w,z);
            plot(z,T); hold on
            plot(z,F,'g');
            xlabel('x');
            ylabel('y');
            legend('Pn(x)','f(x)');
            title(['n = ' num2str(n)]);
            hold off
                x(j)=-5+(j-1)*(10/n);
                y(j)=1/(1+(x(j))^2);
            end
            
            \end{lstlisting}
            
            \begin{center}
                    \includegraphics[scale=0.6]{runge4_a.png}
                    \includegraphics[scale=0.6]{runge8_a.png}
                    \includegraphics[scale=0.6]{runge12_a.png}
            \end{center}
            
            \item
            \begin{lstlisting}
            function T = runge2(n)
            % Computer code for finding the weights of the Barycentric Formula
            % using Chebyshev points
            % Input:  x --- input points;
            % Output: T --- vector of weights to plug into barycentric.m
            %
            % Author: Raghav Thirumulu, Perm 3499720
            % Date:   07/11/2018

            % Create row vectors based on n or number of nodes
            x=zeros(1,n+1);
            y=zeros(1,n+1);
            w=zeros(1,n+1);

            % Iterate through the function, finding the x and y values based on the
            % given formula
            for j=1:n+1
                x(j)=5*cos((n+1-j)*pi/n);
                y(j)=1/(1+(x(j))^2);
            end

            w(1)=0.5*((-1)^n);
            w(n+1)=0.5;
            for i=2:n
                w(i)=(-1)^(n+1-i);
            end

            % We now will create a function with a very high resolution
            m=1000;
            z=zeros(1,m+1);
            F=zeros(1,m+1);

            % Iterate through the formula given for larger number m
            for k=1:m+1
                z(k)=5*cos((m+1-k)*pi/m);
                F(k)=1/(1+(z(k))^2);
            end

            % Plot approximation versus actual function to visualize accuracy
            T = barycentric(x,y,w,z);
            plot(z,T); hold on
            plot(z,F,'g');
            xlabel('x');
            ylabel('y');
            legend('Pn(x)','f(x)');
            title(['n = ' num2str(n)]);
            hold off
            end
            \end{lstlisting}
            
            \begin{center}
            \includegraphics[scale=0.6]{runge4_b.png}
            \includegraphics[scale=0.6]{runge8_b.png}
            \includegraphics[scale=0.6]{runge12_b.png}
            \includegraphics[scale=0.6]{runge100_b.png}
            \end{center}
            
            \item
            \begin{lstlisting}
            function T = runge3(n)
            % Computer code for finding the weights of the Barycentric Formula
            % using equidistributed nodes. We are approximating a different function
            % than from part a
            % Input:  x --- input points;
            % Output: T --- vector of weights to plug into barycentric.m
            %
            % Author: Raghav Thirumulu, Perm 3499720
            % Date:   07/11/2018

            % Create row vectors based on n or number of nodes
            x=zeros(1,n+1);
            y=zeros(1,n+1);
            w=zeros(1,n+1);

            % Iterate through the function, finding the x and y values based on the
            % given formula
            for j=1:n+1
                x(j)=-5+(j-1)*(10/n);
                y(j)=exp(-(x(j)^2)/5);
            end

            % For equidistributed nodes we can use binomial coefficient
            for i=1:n+1
                w(i)=((-1)^(i-1))*nchoosek(n,i-1);
            end

            % We now will create a function with a very high resolution
            m=1000;
            z=zeros(1,m+1);
            F=zeros(1,m+1);

            % Iterate through the formula given for larger number m
            for k=1:m+1
                z(k)=-5+(k-1)*(10/m);
                F(k)=exp(-(z(k)^2)/5);
            end

            % Plot approximation versus actual function to visualize accuracy
            T = barycentric(x,y,w,z);
            plot(z,T); hold on
            plot(z,F,'g');
            xlabel('x');
            ylabel('y');
            legend('Pn(x)','f(x)');
            title(['n = ' num2str(n)]);
            hold off
                x(j)=-5+(j-1)*(10/n);
                y(j)=1/(1+(x(j))^2);
            end
            \end{lstlisting}
            
            \begin{center}
            \includegraphics[scale=0.6]{runge4_c.png}
            \includegraphics[scale=0.6]{runge8_c.png}
            \includegraphics[scale=0.6]{runge12_c.png}
            \end{center}
            
            As we increase n, the approximations using the Barycentric Formula converge to the exact function. Approximations using Chebshev points tend to be more accurate.
        \end{enumerate}
\end{enumerate}


\end{document}
