\documentclass{article}

% Packages
\usepackage{amsmath, amssymb}   % math formatting & symbols
\usepackage{graphicx}           % insert graphics
%\usepackage{eulervm, bookman}   % fonts for math & symbols
\usepackage{fullpage}          % fullpage margins

\begin{document}
% Title 
\title{Math 104A: Homework 2}
\author{Raghav Thirumulu, Perm 3499720 \\ \texttt{rrajuthirumulu@umail.ucsb.edu}}
\maketitle

\begin{enumerate}
    \item % 1 
        \begin{enumerate}
            \item % 1a
                We must construct the basis polynomials first.
                \begin{align*}
                    L_{2,0}(x)    &= \frac{(x-x_{1})(x-x_{2})}{(x_{0}-x_{1})(x_{0}-x_{2})} =                           \frac{(x-1)(x-3)}{(0-1)(0-3)} =
                                     \frac{x^2-4x+3}{3}   \\
                    L_{2,1}(x)    &= \frac{(x-x_{0})(x-x_{2})}{(x_{1}-x_{0})(x_{1}-x_{2})} =                           \frac{(x-0)(x-3)}{(1-0)(1-3)} =
                                     \frac{x^2-3x}{-2}   \\
                    L_{2,2}(x)    &= \frac{(x-x_{0})(x-x_{1})}{(x_{2}-x_{0})(x_{2}-x_{1})} =                           \frac{(x-0)(x-1)}{(3-0)(3-1)} =
                                     \frac{x^2-x}{6}   \\
                \end{align*}    
                Now use the Lagrange interpolation formula to find \textit{P}\textsubscript{2}(\textit{x})
                \begin{align*}
                    P_{2}(x)  & = f(x_{0})L_{2,0}(x) + f(x_{1})L_{2,1}(x) + f(x_{2})L_{2,2}(x)\\
                              & = 1(\frac{x^2-4x+3}{3}) + 1(\frac{x^2-3x}{-2}) - 5(\frac{x^2-x}{6})\\
                              & = -x^2 + x + 1\\
                \end{align*}
            \item % 1b
                Solve for \textit{P}\textsubscript{2}(\textit{x}) where \textit{x} = 2
                \begin{align*}
                    P_{2}(2)  & = -(2)^2 + 2 + 1 = -1\\
                \end{align*}
        \end{enumerate}
    \item % 2 
        Second question goes here
        \begin{enumerate}
            \item % 2a
                two a goes here
            \item % 2b 
                two b goes here
        \end{enumerate}
    \item % 3
        Third question here
        \begin{center}
            \includegraphics[scale=0.5]{flowline1.png}
        \end{center}
    \item % 4 
        Fourth question goes here
\end{enumerate}


\end{document}
