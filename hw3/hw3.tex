\documentclass{article}

% Packages
\usepackage{amsmath, amssymb}   % math formatting & symbols
\usepackage{graphicx}           % insert graphics
\usepackage{listings}
%\usepackage{eulervm, bookman}   % fonts for math & symbols
\usepackage{fullpage}          % fullpage margins

\lstset{language=Matlab,
    basicstyle=\scriptsize\ttfamily,
    stringstyle=\ttfamily,
    showstringspaces=false,
    breaklines=true,
    frameround=ffff,
    frame=single,
}

\begin{document}
% Title 
\title{Math 104A: Homework 3}
\author{Raghav Thirumulu, Perm 3499720 \\ \texttt{rrajuthirumulu@umail.ucsb.edu}}
\maketitle

\begin{enumerate}
    \item % 1 
        \begin{enumerate}
            \item % 1a
            We know that 
            \begin{align*}
                P_{1}(x) = \frac{(x-x_{1})}{(x_{0}-x_{1}}f(x_{0}) + \frac{(x-x_{1})}{(x_{1}-x_{0}}f(x_{1})
            \end{align*}
            We can calculate the error of $ P_{1}(x) $ through the following equation:
            \begin{align*}
                |f(x)-P_{1}(x)| &= |\frac{(x-x_{0})(x-x_{1})}{2}f''(x)| \\
                &= |\frac{(x-x_{0})(x-x_{1})}{2}||f''(x)|
            \end{align*}            
            The maximum value of $ (x-x_{0})(x-x_{1}) $ is at $ x = \frac{x_{0}+x_{1}}{2} $. So
            \begin{align*}
                |(x-x_{0})(x-x_{1})| &\leq \frac{(x_{1}-x_{0})^2}{4}
            \end{align*}
            Also $ |f''(x)| \leq M_{2}$. So we can now write 
            \begin{align*}
                |f(x)-P_{1}(x)| &\leq \frac{1}{2}\frac{(x_{1}-x_{0})^2}{4}M_{2} \\
                &\leq \frac{1}{8}{(x_{1}-x_{0})^2}M_{2}
            \end{align*}
            as desired
            \item % 1b
            Given, $ f(x) = \sin{x} $ where $ 0 \leq x,x_{0},x_{1} \leq \pi/2 $ we can find the second derivative $ f''(x) = -\sin{x}$. \\
            \\
            From part (a):
            \begin{align*}
                    \| f-P_1 \|_{\infty} &\leq \frac{1}{8}(x_{1}-x_{0})^2M_{2} \\
                    \frac{1}{8}(x_{1}-x_{0})^2| -\sin{x_0} | &\leq \frac{1}{8}(x_{1}-x_{0})^2| -\sin{x_1} | \\
                    0 \leq \| f-P_1 \|_{\infty} &\leq \frac{1}{8}x(\pi/2)^2 \\
                    0 \leq \| f-P_1 \|_{\infty} &\leq \pi^2/32
            \end{align*}
            Therefore, the maximum error can be $ \pi^2/32 $. The actual error can be found through the following calculation:
            \begin{align*}
                Error &= \frac{1}{8}(\pi/2)^2(\sin{\frac{\pi}{4}}) \\
                    & = \frac{\pi^2}{32\sqrt{2}}
            \end{align*}
            This error fits the bounds that we calculated. 
        \end{enumerate}
    \item % 2 
        \begin{enumerate}
            \item % 2
            Let
                \begin{align*}
                    w_{n}(x) &= (x-x_{0})(x-x_{1})...(x-x_{n-1})
                \end{align*}
            We can express the polynomial \textit{P}\textsubscript{\textit{n}}(\textit{x}) by
                \begin{align*}
                    P_{n}(x) &= P_{n-1}(x) + c_{n}w_{n}(x) : c_{n} \in \mathbb{R} 
                \end{align*}
            Now let \textit{x} = \textit{x\textsubscript{i}} for some \textit{i} $ \in $ $ {0,1,...,\textit{n}-1} $. Therefore
                \begin{align*}
                    w_{n}(x_{i}) &= 0
                \end{align*}
            and
                \begin{align*}
                    P_{n}(x_{i}) &= P_{n-1}(x_{i}) = f(x_{i})
                \end{align*}
            We can obtain \textit{c\textsubscript{n}} from the fact that \textit{P\textsubscript{n}}(\textit{x\textsubscript{n}}) = \textit{f}(\textit{x\textsubscript{n}}). So
                \begin{align*}
                    f(x_{n}) &= P_{n-1}(x_{n}) + c_{n}w_{n}(x_{n}) \\
                    c_{n} &= \frac{f(x_{n}) - P_{n-1}(x_{n})}{w_{n}(x_{n})}
                \end{align*}
            We also know that \textit{P}\textsubscript{0}(\textit{x}) = \textit{f}(\textit{x}\textsubscript{0}). So we obtain
                \begin{align*}
                    P_{n}(x) &= \sum_{k=0}^{n} c_{k}w_k(x) 
                \end{align*}
            From the formula for Langrange's interpolating polynomial, we say
                \begin{align*}
                    f[x_{0},x_{1},...,x_{k}] &= \frac{f[x_{0},x_{1},...,x_{k}] - f[x_{0},x_{1},...,x_{k-1}]}{x_k-x_{0}}
                \end{align*}
            We can now deduce that
                \begin{align*}
                    P_{n+1}(x) &= P_{n}(x) + f[x_{0},x_{1},...,x_{n+1}]w_{n+1}(x)
                \end{align*}
            and
                \begin{align*}
                    w_{n+1}(x) &= (x-x_{n})w_{n}(x)
                \end{align*}
            The formula for approximating \textit{P\textsubscript{n}} using Lagrange's form is
                \begin{align*}
                    P_{n}(x) &= \sum_{j=0}^{n} f(x_{j}) \frac{x-x_{k}}{\prod_{k=0}_{k \neq j}^{n}(x_{j}-x_{k})}
                \end{align*}
            The coefficient of each term solves to
                \begin{align*}
                    f[x_{0},x_{1},...,x_{n}] &= \sum_{j=0}^{n} \frac{f(x_{j})}{\prod_{k=0}_{k \neq j}^{n}(x_{j}-x_{k})}
                \end{align*}
            as desired. 
            \item % 2b 
                We can see from the procedure above that \textit{P\textsubscript{k}} is unaffected by the ordering of the various \textit{x} values (multiplication is commutative). Therefore, we can conclude that the divivded difference is symmetric in its arguments. 
        \end{enumerate}
    \item % 3
        \begin{enumerate}
            \item % 3a
            \begin{lstlisting}
            function T = newton(x,y,xbar,n)
            % Computer code for evaluating the Newton polynomial based on a set of
            % given points and degree
            % Input:  x    --- Sample x points for evaluating polynomial
            %         y    --- Sample y points for evaluating polynomial
            %         xbar --- Point we want to evaluate at
            %         n    --- degree of Newton polynomial
            %
            % Author: Raghav Thirumulu, Perm 3499720
            % Date:   07/10/2018

            % Create array of n+1 entries for computing values of c
            c=zeros(1,n+1);
            T=0;

            % Iterate through, computing divided difference for each value
            % We use n+1 because matrix values begin with 1 in MATLAB
            for i=1:(n+1)
                f = 1;
                for j=1:(n+1)
                    % We do not want a denominator of 0
                    if j~=i
                        f = f * (xbar-x(j))/(x(i)-x(j));
                    end
                end
                % Sum each term to find the approximation
                T = T + y(i)*f;
            end
            end
            \end{lstlisting}
            \begin{center}
            \includegraphics[scale=0.6]{3arun.PNG}
            \end{center}
            
            \item % 3b
            \begin{lstlisting}
            function T = plot_newton_error()
            % Computer code for plotting the error of our Newton interpolation
            % using equidistributed nodes.
            % Input:  None
            % Output: Plot of f(x), P10(x), and error
            %
            % Author: Raghav Thirumulu, Perm 3499720
            % Date:   07/11/2018

            % Create row vectors for storing values of x and f(x) for evaluating
            % P10(x)
            x=zeros(1,11);
            y=zeros(1,11);

            % Use the given function to iterate and store these points
            for j=1:11
                x(j)=-1+(j-1)*(2/10);
                y(j)=exp((-1)*((x(j))^2));
            end

            % Create row vectors because we want to solve P10(x) for a 101 different x
            % values using the same equidistributed node equation
            xbar=zeros(1,101);
            f=zeros(1,101);
            err=zeros(1,101);
            for i=1:101
                xbar(i)=-1+(i-1)*(2/100);
                % Evaluate Newton polynomial at each of the 101 different x points and
                % solve for the error at each point using the given equation
                T(i) = newton(x,y,xbar(i),10);
                f(i)=exp((-1)*(xbar(i)^2));
                err(i) = abs(T(i)-f(i));
            end

            figure(1);
            plot(xbar,T); hold on
            plot(xbar,f,'r');
            xlabel('x');
            ylabel('y');
            legend('P10(x)','f(x)');
            hold off

            figure(2);
            plot(xbar,err);
            xlabel('x');
            ylabel('Error ( f(x) - P10(x) )');
            end
            \end{lstlisting}
            \begin{center}
                \includegraphics[scale=0.6]{3b1.png}
                \includegraphics[scale=0.6]{3b2.png}
            \end{center}

        \end{enumerate}
    \item % 4 
        Given: \textit{f}(0)=0, \textit{f}'(0)=0, \textit{f}(1)=2, \textit{f}'(1)=3. Let 
        \begin{align*}
            f(x) &= a_{0} + a_{1}x + a_{2}x^2 + a_{3}x^3 \\
            f'(x) &= a_{1} + 2a_{2}x + 3a_{3}x^2 
        \end{align*}    
        Plugging in the given conditions into the equations above gives us
        \begin{align*}
            a_{0} &= 0 \\
            a_{1} &= 0 \\
            a_{0} + a_{1} + a_{2} + a_{3} &= 2 \\
            a_{1} + 2a_{2} + 3a_{3} &= 3 
        \end{align*}
        Through elimination and substitution, we solve for \textit{a}\textsubscript{2} and \textit{a}\textsubscript{3}.
        \begin{align*}
            a_{2} &= -3 \\
            a_{3} &= 5
        \end{align*}
        Plugging back values of \textit{a} into the equation gives us the interpolating polynomial:
        \begin{align*}
            f(x) = 5x^3 -3x^2
        \end{align*}

\end{enumerate}
\end{document}
